% Abstract file

% split into two sections of abstract and declaration

\newpage
%\makeatletter
%\vspace*{-50\p@} % makeschapterhead starts with a vspace of 50pt
%\@makeschapterhead{Summary and declaration}
\chapter*{Declaration}
%\makeatother
This thesis presents upper limit results on the strength of gravitational waves from known neutron
stars using data from the LIGO and \geo interferometric \gw detectors as part of the LIGO Scientific
Collaboration (LSC). This work was performed as part of the LSC, and as such much work has been done
by a great many people in building and maintaining the detectors, obtaining and characterising the
data, and developing software components that we have used. Below I provide a summary of the work
performed by myself with reference to that of others.

The first chapter describes the basic theory behind gravitational radiation and summarises some of
the properties of \gw sources. A brief description is given of interferometric \gw detectors,
including their major noise sources. A summary of the previous searches performed by the LSC is
given along with some of the main results.

Chapter 2 describes neutron stars and their potential as sources of continuous gravitational
waves. A description of the search algorithm and Bayesian parameter estimation technique, as
developed by Dupuis and Woan \cite{DupuisWoan:2005}, is provided. New to this thesis are
descriptions of techniques to deal with pulsar timing noise and pulsars in binary systems. These
have been developed by myself and Graham Woan, with the software being validated and checked by
myself. The binary software has made use of the pulsar timing software \tempo \cite{TEMPO}.
Verification of the code has relied greatly on information provided to us by Michael Kramer of
Jodrell Bank Observatory.

In Chapter 3 the results of searches for \gws from 93 known pulsars using data from the LSC S3
and S4 science runs are presented. Many of the pulsar timing parameters were supplied by Michael
Kramer and Andrew Lyne of Jodrell Bank Observatory, and the Australia Telescope National Facility
pulsar catalogue. I describe the selection of pulsars used in the analysis, noting errors on the
pulsars parameters and for the first time discussing the possible effects of timing noise on our
analysis for all pulsars. Results for 65 of these pulsars have never before been obtained. The S3
and S4 analyses have been performed by myself and include the new method of combining the two data
sets for a single analysis, with all the results, in terms of upper limits on \gw amplitude and
pulsar ellipticity, presented here. I suggest a method of displaying some results on a moment of
inertia-ellipticity plane, thus providing an exclusion region in this parameter space. 

In Chapter 4 the emission of gravitational waves from quasi-normal modes ring-downs of neutron
stars is discussed. Also discussed is the generation of such modes from neutron star glitches. I
present a preliminary search for \gw ring-down signals from the magnetar SGR\,1806-20 during a
giant flare on 27$^{\rm th}$ December 2004. For our search I describe and make use of matched
filtering software developed by Jolien Creighton and others, but which had otherwise never been used
for neutron star ring-down searches. I also describe a search method developed by myself and
Graham Woan based on Bayesian evidence, by making use work of Bretthorst \cite{Bretthorst:1988}.

Chapter 5 describes the future of both these searches and prospects for the latest science run.

\newpage

\chapter*{Abstract}
We have used data from the third and fourth science runs of the laser interferometric
gravitational wave detectors LIGO and \geo to produce upper limits on the emission of \gws from a
selection of known neutron stars. Two different emission mechanisms are looked into; i) the
emission of continuous \gws from triaxial neutron stars; and ii) emission of quasi-normal mode
ring-downs from glitching neutron stars.

We have produced upper limits on the gravitational wave amplitude and ellipticity for 93 known
pulsars assuming continuous emission via triaxiality. This selection of pulsars includes the
majority of currently known pulsars with frequencies $> 25$\,Hz, with many within binary systems and
globular clusters. New algorithms to take into account the motions within binary systems and
possible effects of pulsar timing noise are presented. Also shown is the first analysis to combine
the data sets from two distinct science runs as a method of lowering the upper limits. The results
are starting to push into the range of plausible neutron star ellipticities, with the Crab pulsar
closely approaching the limit that can be set through spin-down arguments. For the 32 of these
pulsars in globular clusters the results provide upper limits independent of the cluster dynamics.
The astrophysical significance of these results is discussed. Along with results from true pulsars
we also present the extraction of simulated signals injected into the interferometers during the
science runs. These provide validation checks of both the extraction software and the coherence of
the detectors.

Two techniques are discussed in relation to searching for quasi-normal mode ring-down signals from
excited neutron stars, for example during a glitch; one based on matched filtering and the other
based on Bayesian evidence. These are both applied to a search for such a signal from SGR\,1806-20
during a GRB on 27$^{\rm th}$ December 2004, using the LIGO H1 detector and \geo data. This search
provided upper limits on the energy released in \gws via quasi-normal modes over the range of
frequencies from 1-4\,kHz. These are compared with results from a previous search using the bar
detector AURIGA \cite{Baggio:2005} and theoretical arguments. The limitations of the search and
search techniques, and possible extensions to these, are discussed.

The future of these searches is discussed with regard to extensions to the analysis techniques and
number of potential sources. Particular emphasis is placed on searches using data from the current
LSC S5 science run.
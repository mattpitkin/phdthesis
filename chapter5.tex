%%%%%%%%%%%%%%%%%%%%%%%%%%%%%%%%%%%%%%%%%%%%%%%%%%%%%%%%%%%%%
% Matt Pitkin 26/04/05                                      %
%                                                           %
% Chapter 5: Give me a "C"! Give me an "O", etc etc         %
%%%%%%%%%%%%%%%%%%%%%%%%%%%%%%%%%%%%%%%%%%%%%%%%%%%%%%%%%%%%%

%\chapquote{Perfection is the destination, imperfection is the journey}{Derek Smalls - This is
%Spinal Tap}
\chapquote{You win again, gravity.}{Captain Zapp Brannigan - Futurama}

\chapter{Future work}
The present status of the ongoing search for \gws is that no evidence has yet been seen for their
direct detection. Despite this pessimistic sounding statement we are still advancing ever closer
towards the first direct sighting. In the meantime we are reaching the point where interesting
astrophysics can be gained from our null results. Upper limits on various emission mechanisms and
event rates can begin to constrain theoretical models of sources and population studies.

This thesis has given the current status of the search for \gws from a selection of known neutron
stars via two different mechanisms. This, however, is not the end state of each search, with much
more work continuing in the future. 

\section{The known pulsar search}
One of the first things to be noted is that here we have searched for 93 known pulsars, this being
the number for which adequate timing solutions were available. There are currently 150 pulsars
within our band, with more being discovered from radio surveys on a regular basis. This will
increase our search sample and hopefully provide some candidates with more promising detection
possibilities i.e. young pulsars with high spin-down rates. We are in close contact with radio
astronomers to get the most up-to-date timing solutions possible, although obviously timing
observations from the time of a science run can only be used in a post-run analysis. As we have
seen from their inferred spin-down upper limits the {\it known} radio pulsars are not the best
candidates for \gw detection. There are, however, several pulsars only seen in X-rays which
might provide better candidates, for example PSR\,J0537-6910 described in
\S\ref{PSRJ0537-6910section}. With several space-based high energy telescopes (the {\it Rossi X-ray
Timing Explorer}, {\it Chandra}, {\it XMM-Newton} and {\it INTEGRAL}) currently operating, X-ray
pulsars, mainly those in LMXB systems, are becoming a far more studied source. These provide more
enticing candidates, with conditions for sustaining \gw emission being more favourable. Other
enticing places to search are SNRs, with Veitch {\it et al.} (2005) \cite{Veitch:2005} developing a
search for a possible remnant of SN\,1987A. Due to the many X-ray pulsars and other potential
sources having far less well defined, or unknown, parameters than many radio pulsars, a search as
performed in this thesis would be inadequate. The MCMC search for \gws from a SN\,1987A remnant in
\cite{Veitch:2005} could, for instance, be extended to search over binary parameters and used in
LMXB searches.

% refinements to heterodyne algorithm
The search algorithm, as it is currently used by performing a single fine grained heterodyne at the
exact pulsar phase, is computationally fairly slow. The main speed restriction on this is having to
compute the Doppler correction to the SSB (and binary system barycentre for some) for every data
sample. This will only get slower as more pulsars are included. The speed of the algorithm also
becomes a problem if the data has to be re-analysed many times, for example if certain data segments
were missed or new calibration data is used. One possible method to reduce the computational burden
of the search, and more easily allow repeated analysis, would be to return to a two stage heterodyne
process similar to that used in the S1 pulsar search \cite{Abbott:2004}. This performed an initial
heterodyne at the pulsar frequency, but did not include spin-down or Doppler corrections, allowing
the data to be massively down-sampled and filtered before the finer corrections were applied in a
second heterodyne. There are some limitations on this, in that the down-sampling and filtering must
be able to accommodate the frequency range drift caused by Doppler motions, with particular care for
binary pulsars, although going from 16\,384\,Hz to 1\,Hz would still provide plenty of range. In
practice a less crude initial heterodyne using as many parameters as possible can also be used. This
would mean that during the course of a science run the initial heterodyne can still be performed
using older timing solutions, and then the second stage heterodyne used to perform additional phase
corrections when up-to-date timing information is made available - this is essentially what is done
for the Crab pulsar with regards to timing noise corrections. 

The heterodyne approach may well be phased out in the longer term. When there are many pulsars
across a wide range of frequencies a more sensible approach may be to use Fourier transforms of the
data, which essentially provide a fixed frequency heterodyne over all frequencies. Short time
baseline Fourier transforms (SFTs) of the data for LIGO and \geo are already produced for the
frequency domain pulsar searches and could be used for our purposes. The SFTs need to be short
enough that the source's signal is not spread out over many frequency bins due to Doppler/spin-down
effects. Under such an approach the exact frequency of the source would be calculated and
extrapolated between successive frequency bins. This cuts down the problem computationally, as
the SFTs are pre-produced and the frequency only needs to be calculated at the rate of the time
baseline of the SFTs.

% refinements to marginalisation - analytical or approximations to analytical solutions
The marginalisation in equation~\ref{margoverall} used in the Bayesian parameter estimation
currently numerically sums over the nuisance parameters. The grid over which the marginalisation
takes place is limited by computer memory restraints, so the integration is only approximate. The
possibility of performing these integrals analytically needs to be explored.

\section{The ring-down search}
The ring-down search presented in this thesis is in a very preliminary state and was more about
outlining and testing the algorithms than producing a full and thorough search pipeline. Much of
what needs to be done is outlined at the end of Chapter~4. The particular case of the search for a
signal from the $27^{\rm th}$ December GRB needs a more thorough study of event background rates
and the coincidence analysis. When the evidence based search becomes more stable against spectral
lines it will be provide a good complementary strategy to the matched filtering search. There are
other glitches from the Crab and Vela pulsars to be searched for.

With S5 starting and providing an almost continuous data set for a year or so, it provides the
opportunity to catch as many glitches as we can. With pulsars such as the Crab, Vela and J0537-6910
being prolific glitchers there should be several events during the run. As with the known pulsar
continuous wave search, the timing for these glitches can only be obtained post-event. This will
again need close cooperation with those observing the pulsars to obtain accurate information as
soon as possible. Glitches seen in accreting X-ray pulsars provide an excellent target, with
fractional frequency changes seen up to $\Delta\nu/\nu \sim 3\ee{-4}$ for one such object
(SAX~J2103.5+4545) \cite{Stark:2004} being two orders of magnitude above the maximum seen in Vela
glitches. Another good accreting X-ray potential target is KS 1947+300 which had a glitch of
fractional frequency change at $\Delta\nu/\nu \sim 4\ee{-5}$ \cite{Galloway:2004}.

\section{S5 and beyond}
The fifth science run (S5) of the LSC interferometers started officially on $4^{\rm th}$ November
2005, with H1 and H2 to start with, and L1 and \geo joining later. This run marks the start of full
time operation of the interferometers at approximately their design sensitivities. This should give
sensitivities to pulsars at around that given in figure~\ref{h0results}, and allow us to beat the
spin-down upper limits for at least the Crab pulsar by approximately an order of magnitude.

Towards the end of the decade LIGO will be decommissioned and upgrades to Advanced LIGO installed.
This should give access to many more potential sources and beat the spin-down limits for many
pulsars. Possible upgrades to \geo to tune it to the high frequencies could provide a good window
to look at oscillation modes of neutron stars, with possibilities extending beyond the fundamental
$f$-mode. This would perhaps be able to spot high frequency oscillations from newborn neutron stars
to a large distance.

Hopefully \gw astronomy will soon be able to provide much needed insight into the structure and
nature of neutron stars, which is currently open to much speculation. This will be complimentary
to electromagnetic studies, but should provide a wealth of unique information.